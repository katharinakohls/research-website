\documentclass{article}
\usepackage{hyperref}
\usepackage[utf8]{inputenc}
\usepackage{parskip}
\usepackage{enumitem}
\usepackage{hyperref}
\usepackage{tcolorbox}
\usepackage{geometry}
 \geometry{
 a4paper,
 total={170mm,257mm},
 left=20mm,
 top=20mm,
 }
 \usepackage{graphicx}
 \usepackage{titling}
 \setlength\parindent{0pt}
 \title{Thesis proposal\\
 \large Name of your thesis}
\author{Your name}
\date{\today}

\usepackage{lipsum}  
\usepackage{cmbright}

\begin{document}

\maketitle

\section*{Supervision}
\begin{tabular}{ l l }
	First supervisor: & Prof. Dr. Katharina Kohls \\ 
	Second supervisor: & -  \\  
\end{tabular}

\section*{Motivation}
The motivation section should explain \textbf{why} the topic is relevant, timely, and interesting. This is where you convince the reader of the importance of your research.

\textbf{Questions to address:}
\begin{itemize}[label=--]
	\item What problem or challenge does your topic address?
  	\item What is the real-world context of your problem?
	\item Why is this problem worth investigating?
	\item What are the practical or theoretical benefits of solving it?
	\item Is there a gap in existing research or practice?
\end{itemize}

The motivation serves as an introduction to your proposal. It should already contain a few references to important related work and provide a full argumentation towards the relevance of the project.

\section*{Related Work}
In this section, provide a short literature review of key work related to your topic. Show that you are aware of the research landscape and identify how your work will contribute.

\textbf{Questions to address:}
\begin{itemize}[label=--]
  \item What are the most relevant papers, systems, or approaches?
  \item What have others already done in this area?
  \item How does your work differ or extend prior work?
  \item Are there any shortcomings or gaps in the existing literature?
\end{itemize}

For every reference or group of references provide a connection to tour own work. Discuss what they have in common and how they differ from the intended approach in your thesis. After these smaller discussions, also prvoide a short paragraph that concludes the overview of the current state of the art and once more explains the relevanve of your thesis, now from the perspective of literature.

\section*{Research Question}
This is the core of your proposal. Clearly state your main research question and break it down into smaller, manageable sub-questions. Your whole thesis should aim to answer these questions.

\textbf{Questions to address:}
\begin{itemize}[label=--]
  \item What is your main research question?
  \item Is it specific, clear, and researchable?
  \item What sub-questions need to be answered to address the main question?
  \item How do these sub-questions relate to one another?
\end{itemize}

Make sure to justify and explain your research questions. At this point your already have the motivation and related work as backup. The research question should be a logical consequence of everything you discussed before. The sub questions help to structure your thesis.

\section*{Timeline}
Outline a rough timeline that shows how you plan to approach and complete your thesis. This helps both you and your advisor manage expectations and track progress.

\textbf{Questions to address:}
\begin{itemize}[label=--]
  \item What are the major milestones (e.g., literature review, prototyping, data collection, writing)?
  \item When do you plan to complete each of these steps?
  \item Are there any dependencies or risks you need to consider?
  \item How will you ensure you stay on track?
\end{itemize}

\vspace{1em}
\begin{tcolorbox}[colback=blue!5!white, colframe=blue!50!black, title=Tip]
Be realistic with your goals and timeline. It's better to promise less and deliver more than the opposite.
\end{tcolorbox}

\bibliographystyle{abbrv}
\bibliography{template}


\end{document}